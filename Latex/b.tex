\documentclass{article}
\usepackage{amsmath}
\usepackage{booktabs} % For better looking tables
\usepackage{geometry} % To change margins (optional)
\geometry{a4paper, margin=1in}

\begin{document}

\title{LaTeX Editor - Algorithm Assignment 2}
\author{Gurram Dhanush \\ 242IS036}
\date{2024-09-23}
\maketitle

\section{System Details}
The experiments were conducted on a single machine with the following specifications:
\begin{itemize}
    \item \textbf{Processor:} 11th Gen Intel(R) Core(TM) i5-11500 @ 2.70GHz
    \item \textbf{Architecture:} x86\_64
    \item \textbf{Operating System:} Ubuntu 22.04 LTS
    \item \textbf{Software:} Visual Studio Code
\end{itemize}

\section{Mathematical Equations}
Here are some updated equations:
\begin{enumerate}
    \item The area of a circle:
    \[
    A = \pi r^2
    \]
    \item Newton's second law:
    \[
    F = ma
    \]
    \item The exponential growth equation:
    \[
    N(t) = N_0 e^{rt}
    \]
    \item Fourier series representation:
    \[
    f(x) = a_0 + \sum_{n=1}^{\infty} \left( a_n \cos(nx) + b_n \sin(nx) \right)
    \]
    \item Derivative of sine:
    \[
    \frac{d}{dx}\sin x = \cos x
    \]
    \item The chain rule:
    \[
    \frac{dy}{dx} = \frac{dy}{du} \cdot \frac{du}{dx}
    \]
    \item Taylor series expansion:
    \[
    f(x) = f(a) + f'(a)(x - a) + \frac{f''(a)}{2!}(x - a)^2 + \dots
    \]
    \item Volume of a sphere:
    \[
    V = \frac{4}{3} \pi r^3
    \]
    \item 2x2 Matrix:
    \[
    B = \begin{pmatrix}
    1 & 2 \\
    3 & 4
    \end{pmatrix}
    \]


\section{Table: Straw Hats Characters}
Here is a table of characters from Straw Hats:
\begin{table}[h]
    \centering
    \caption{Straw Hats Characters}
    \begin{tabular}{@{}lll@{}}
        \toprule
        \textbf{Name} & \textbf{Occupation}      & \textbf{Age} \\ \midrule
        Luffy         & Captain                     & 29           \\
        Zoro          & Right Hand                  & 30           \\
        Nami          & Ship Navigator              & 25           \\
        Usopp         & Sniper                      & 25           \\
        Sanji         & Chef                        & 24           \\ 
        Chopper       & Doctor                      & 40           \\ \bottomrule
    \end{tabular}
\end{table}
\end{enumerate}
\end{document}
